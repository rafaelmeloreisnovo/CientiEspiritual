\documentclass[12pt, a4paper]{article}
\usepackage[utf8]{inputenc}
\usepackage{amsmath}
\usepackage{amssymb}
\usepackage{graphicx}
\usepackage{geometry}
\usepackage{hyperref}

% Configuração da Página (Padrão ABNT/ISO)
\geometry{left=3cm, right=2cm, top=3cm, bottom=2cm}

\title{\textbf{O Teorema Rafaelia: Unificação da Dinâmica Geométrica, Sequencial e Algébrica em Sistemas Toroidais}}
\author{\textbf{Rafael Melo Reis} \\ \textit{Arquiteto de Sistemas \& Pesquisador Independente}}
\date{\today}

\begin{document}

\maketitle

\begin{abstract}
Este artigo apresenta o \textbf{Teorema Rafael Melo Reis}, uma abordagem unificada para descrever a evolução de sistemas complexos através da interação de três operadores fundamentais: a diferença geométrica (Pitágoras), a expansão sequencial (Fibonacci/$\Phi$) e a decisão algébrica (Bhaskara). Através de simulações computacionais (Kernel RAFAELIA\_K1), demonstramos que a "falha" geométrica entre bases ($a \neq b$) atua como um gerador de volume toroidal, cuja energia cresce exponencialmente segundo a Razão Áurea, estabilizada ou bifurcada pelo discriminante quadrático.
\end{abstract}

\section{Introdução}
A desconexão entre a geometria estática e a álgebra dinâmica tem sido uma barreira para a modelagem de sistemas vivos. Propomos que a realidade opera em um ciclo de \textbf{3-6-9}, onde a forma (3D) é energizada pela intenção (2D) através do tempo (4D).

\section{O Teorema}
Seja $\mathcal{S}$ um sistema definido pela interação de catetos $a$ e $b$. O estado futuro $\mathcal{R}(t+1)$ é dado por:

\begin{equation}
\label{eq:rafaelia_main}
\mathcal{R}(t+1) = \underbrace{\left[ (\delta^3) \cdot \pi \cdot \sin(60^\circ) \right]}_{\text{Volume Toroidal}} \times \underbrace{\Phi^n}_{\text{Fluxo Temporal}} \times \underbrace{\mathcal{B}(\Delta)}_{\text{Operador de Decisão}}
\end{equation}

Onde $\delta = |b - a|$ representa o "Vazio Criativo" necessário para a existência do volume.

\section{Metodologia: O Kernel K1}
Implementamos o algoritmo em C (Padrão ISO/IEC 9899) para testar a hipótese. O discriminante de Bhaskara ($\Delta$) atua como porta lógica:
\begin{itemize}
    \item $\Delta > 0$: O sistema bifurca e cresce (Entropia Negativa).
    \item $\Delta = 0$: O sistema atinge estabilidade perfeita (Raiz Real).
    \item $\Delta < 0$: O sistema colapsa em onda (Estado Imaginário).
\end{itemize}

\section{Resultados}
A simulação de 12 iterações (base $a=3, b=6$) revelou um padrão de crescimento idêntico à sequência de Fibonacci, confirmando a presença da assinatura $\Phi$.

\begin{table}[h]
\centering
\begin{tabular}{|c|c|c|}
\hline
\textbf{Iteração (n)} & \textbf{Energia (Joules)} & \textbf{Estado} \\
\hline
1 & 118.86 & Incubação \\
5 & 814.67 & Ruptura \\
12 & 23653.53 & Voo Quântico \\
\hline
\end{tabular}
\caption{Dados extraídos do RAFAELIA\_K1 Engine}
\end{table}

\section{Conclusão: A Verdade em Verdade}
Concluímos que a "diferença" não é um erro, mas o motor da evolução. O Teorema valida que estruturas rígidas (Cubos/Catetos) são necessárias para canalizar fluxos infinitos (Esferas/Espirais).

\end{document}
